\chapter{Introduction}

This document is intended to be both an example of the HTWG Konstanz \LaTeX{} template for reports and theses, as well as a short introduction to its use. It is not intended to be a general introduction to \LaTeX{} itself,\footnote{We recommend \url{http://en.wikibooks.org/wiki/LaTeX} as a reference and a starting point for new users.} and we will assume the reader to be familiar with the basics of creating and compiling documents.


\section{Document Structure}

Since a report, and especially a thesis, might be a substantial document, it is convenient to break it up into smaller pieces. In this template we therefore give every chapter its own file. The chapters (and appendices) are gathered together in \texttt{report.tex}, which is the master file describing the overall structure of the document. \texttt{report.tex} starts with the line
\begin{quote}
    \texttt{\textbackslash documentclass\{htwg-report\}}
\end{quote}
which loads the HTWG Konstanz report template. The template is based on the \LaTeX{} \texttt{book} document class and stored in \texttt{tudelft-report.cls}. The document class accepts several comma-separated options. The default language is English, but this can be changed to Dutch (\emph{e.g.}, for bachelor theses) by specifying the \texttt{dutch} option:
\begin{quote}
    \texttt{\textbackslash documentclass[german]\{htwg-report\}}
\end{quote}
Furthermore, hyperlinks are shown in blue, which is convenient when reader the report on a computer, but can be expensive when printing. They can be turned black with the \texttt{print} option. This will also turn the headers black instead of cyan.

If the document becomes large, it is easy to miss warnings about the layout in the \LaTeX{} output. In order to locate problem areas, add the \texttt{draft} option to the \texttt{\textbackslash documentclass} line. This will display a vertical bar in the margins next to the paragraphs that require attention. Finally, the \texttt{nativefonts} option can be used to override the automatic font selection (see below).

This template has the option to automatically generate a cover page with the \texttt{\textbackslash makecover} command. See the next section for a detailed description.

The contents of the report are included between the \texttt{\textbackslash begin\{document\}} and \texttt{\textbackslash end} commands, and split into three parts by
\begin{enumerate}
\item\texttt{\textbackslash frontmatter}, which uses Roman numerals for the page numbers and is used for the title page and the table of contents;
\item\texttt{\textbackslash mainmatter}, which uses Arabic numerals for the page numbers and is the style for the chapters;
\item\texttt{\textbackslash appendix}, which uses letters for the chapter numbers, starting with `A'.
\end{enumerate}
The title page is defined in a separate file, \emph{e.g.}, \texttt{title.tex}, and included verbatim with \texttt{\textbackslash input\{title\}}.\footnote{Note that it is not necessary to specify the file extension.} Additionally, it is possible to include a preface, containing, for example, the acknowledgements. An example can be found in \texttt{preface.tex}. The table of contents is generated automatically with the \texttt{\textbackslash tableofcontents} command. Chapters are included after \texttt{\textbackslash mainmatter} and appendices after \texttt{\textbackslash appendix}. For example, \texttt{\textbackslash input\{chapter-1\}} includes \texttt{chapter-1.tex}, which contains this introduction.

\section{Bibliography}

The bibliography, finally, is generated automatically with
\begin{lstlisting}
\printbibliography[heading=bibintoc]
\end{lstlisting}
from \texttt{bib/report.bib}. The bibliography style is specified in \texttt{htwg-report.cls}. As an example, we cite the paper by Nobel Prize winner Andre Geim and his pet hamster \cite{Geim2001}. If you need to use a different style, change

\begin{lstlisting}
%% BIB
\RequirePackage[
	backend=biber,
	style=alphabetic,
	sorting=ynt
]{biblatex}
\DefineBibliographyStrings{english}{%
  bibliography = {References},
}
\end{lstlisting}

As compiler, use \texttt{biber} to compile and generate the bibliography.

\section{Cover and Title Page}

This template will automatically generate a cover page if you issue the \texttt{\textbackslash makecover} command. However, before generating the cover, you need to provide the information to put on it. This can be done with the following commands:

\begin{lstlisting}
%% 'reporttype' add background elements to the cover / front page
%% possible values are:
%% bachelor	--> B S C
%% master	--> M S C
%% other		--> none
\reporttype{bachelor}

\reporttypetext{Bachelor Thesis}
\end{lstlisting}

\section{Chapters}

Each chapter has its own file. For example, the \LaTeX{} source of this chapter can be found in \texttt{chapter-1.tex}. A chapter starts with the command
\begin{quote}
    \texttt{\textbackslash chapter\{Chapter title\}}
\end{quote}
This starts a new page, prints the chapter number and title and adds a link in the table of contents. If the title is very long, it may be desirable to use a shorter version in the page headers and the table of contents. This can be achieved by specifying the short title in brackets:
\begin{quote}
    \texttt{\textbackslash chapter[Short title]\{Very long title with many words which could not possibly fit on one line\}}
\end{quote}
Unnumbered chapters, such as the preface, can be created with \texttt{\textbackslash chapter*\{Chapter title\}}. Such a chapter will not show up in the table of contents or in the page header. To create a table of contents entry anyway, add
\begin{quote}
    \texttt{\textbackslash addcontentsline\{toc\}\{chapter\}\{Chapter title\}}
\end{quote}
after the \texttt{\textbackslash chapter} command. To print the chapter title in the page header, add
\begin{quote}
    \texttt{\textbackslash setheader\{Chapter title\}}
\end{quote}

Chapters are subdivided into sections, subsections, subsubsections, and, optionally, paragraphs and subparagraphs. All can have a title, but only sections and subsections are numbered. As with chapters, the numbering can be turned off by using \texttt{\textbackslash section*\{\ldots\}} instead of \texttt{\textbackslash section\{\ldots\}}, and similarly for the subsection.
\section{\textbackslash section\{\ldots\}}
\subsection{\textbackslash subsection\{\ldots\}}
\subsubsection{\textbackslash subsubsection\{\ldots\}}
\paragraph{\textbackslash paragraph\{\ldots\}}
Lorem ipsum dolor sit amet, consectetur adipisicing elit, sed do eiusmod tempor incididunt ut labore et dolore magna aliqua. Ut enim ad minim veniam, quis nostrud exercitation ullamco laboris nisi ut aliquip ex ea commodo consequat. Duis aute irure dolor in reprehenderit in voluptate velit esse cillum dolore eu fugiat nulla pariatur. Excepteur sint occaecat cupidatat non proident, sunt in culpa qui officia deserunt mollit anim id est laborum.

\section{Fonts and Colors}

If you want to use the HTWG house style font \texttt{Swiss 721} it is necessary to put the font as \texttt{.ttf} file under \texttt{fonts}. As fallback font \texttt{Arial} is used. For more informations to the HTWG house style font see \url{https://www.htwg-konstanz.de/hochschule/einrichtungen/stabsstelle-kommunikation/corporate-design-logo/}.


