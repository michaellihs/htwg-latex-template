% !TeX spellcheck = en_US
% !TEX root = ../thesis.tex
\thispagestyle{plain}
\addcontentsline{toc}{chapter}{Extended Abstract}
\vspace*{11pt}
\begin{center}
	{\LARGE \textbf{\textsf{Extended Abstract}}}
\end{center}

\bigskip
\begin{center}
	\begin{tabular}{p{3.2cm}p{9.6cm}}
	Title: & \thema \\
	& \\
	Masters candidate: & \autor \\
	& \\
	Supervisor: & \firma \\[1.1ex] & \prueferA  \\[.5ex]
	&  \prueferB \\
	& \\
	Submission date: & \abgabedatum \\
	& \\
	Keywords: & \schlagworte \\
	& \\
	\end{tabular}
\end{center}

\bigskip

\noindent
\textbf{\textsf{Introduction}}
\vspace*{3pt}

Extended Abstract über 2 Seiten. Beispielhafte Texte aus anderen Teamprojekten oder Abschlussarbeiten können aus dem verlinkten Dokument entnommen werden \href{http://www.ios.htwg-konstanz.de/sites/default/files/jb/annualreport17.pdf}{http://www.ios.htwg-konstanz.de/sites/default/files/jb/annualreport17.pdf}.

\noindent Dieser Text soll als Dokumentation des Teamprojekts für den zukünftigen Jahresbericht des Institut für Optische Systeme dienen. Gerne können auch Bilder eingefügt werden. Ebenso wichtig ist es auch die Referenzen aufzulisten wie z.B. \cite{Geim2001}. Die Referenzen werden mit Biber erstellt.

Hier kann man auch zitieren. \cite{ext_sharp}





\bibliographystyle{acm}
\begin{thebibliography}{99}
% hier die Bibliografie ergänzen.
\bibitem{ext_sharp} Sharp, N., and Crane, K. Variational surface cutting. \textit{ACM Trans. Graph. 37,} 4 (2018).
\end{thebibliography}